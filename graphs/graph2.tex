\documentclass[8pt]{article}

\usepackage{fullpage}
\usepackage[margin=.7in]{geometry}
\usepackage{epic}
\usepackage{eepic}
\usepackage{graphicx}
\usepackage{listings}
\usepackage{color}

\begin{document}
\hfill \small{\today} \\
\begin{center}{\bf
\center{CS 537 - Project 2b Graph2}
\center{Rashi Jalan, Mihir Shete}
}\end{center}
\begin{center}{\includegraphics[scale=0.5]{mlfq2.png}} \\
{Figure 1 \footnotesize{[1]}} \\
\end{center}

\textbf{Figure 1} represents the workloads of 2 processes p4 and p5.
Both processes p4 and p5 run an infinite while loop (see \textit{spin()}) and we can see that both of them are downgraded to the lowest priority level and are run in round-robin fashion.

%Stackoverflow http://stackoverflow.com/questions/4439605/c-source-code-in-latex-document
\lstset{
  language=C,                % choose the language of the code
  numbers=left,                   % where to put the line-numbers
  stepnumber=1,                   % the step between two line-numbers.        
  numbersep=5pt,                  % how far the line-numbers are from the code
  backgroundcolor=\color{white},  % choose the background color. 
  showspaces=false,               % show spaces adding particular underscores
  showstringspaces=false,         % underline spaces within strings
  showtabs=false,                 % show tabs within strings adding particular underscores
  tabsize=2,                      % sets default tabsize to 2 spaces
  captionpos=b,                   % sets the caption-position to bottom
  breaklines=true,                % sets automatic line breaking
  breakatwhitespace=true,         % sets if automatic breaks should only happen at whitespace
  fontadjust=true,
  basicstyle=\ttfamily\footnotesize
}

\lstinputlisting{spin2.c}
\vfill
\footnotesize{[1] These graphs were obtained by running tests on a vm with sigle cpu core} \\

\end{document}
